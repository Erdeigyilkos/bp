\documentclass[11pt]{article}

 \usepackage{subfig}		
 \usepackage{tikz}		
 \usepackage{float}
 \usepackage{multicol}
 \usepackage[utf8]{inputenc}
\usepackage[czech]{babel}
\usepackage{graphicx}
\usepackage{graphicx}
\usepackage[margin=1in]{geometry}
\usepackage{blindtext}

 \title{\textbf{PROTOKOL \\ bakalářské aplikace na čítání osob na základě analýzy WiFi~provozu}}
\date{ }

\begin{document}

\maketitle


Tento protokol byl automaticky vygenerován bakalářskou aplikací na čítání osob na základě analýzy WiFi provozu.
Grafy byly vygenerovány dle Vámi zadaných parametrů. 


\vspace{1cm}

\section*{Graf č.1: Počet rozeznaných zařízení dle výrobců v časových intervalech}
Tento graf prezentuje počty rozeznaných zařízení odchycených v uživatelem zvolených časových intervalech, a to dle jednotlivých výrobců. Osa X zobrazuje časové intervaly a osa Y zobrazuje počty odchycených a zároveň rozeznaných zařízení v kusech. 
\vspace{1cm}

\begin{figure}[!h]
  \includegraphics[width=\linewidth]{stackbar.png}

\end{figure}


\newpage
  \section*{Graf č.2: Počet rozeznaných zařízení dle výrobců}
  Tento graf zobrazuje počty zařízení jednotlivých výrobců odchycených bakalářskou aplikací za celou dobu sběru dat. U výseče je uvedeno jméno výrobce, kterého daná výseč znázorňuje a počet jeho odchycených zařízení. V případě, že se Vám vygeneroval výsečový graf znázorňující počet odchycených zařízení u pěti nejčastějších výrobců, byla bakalářskou aplikací odchycena zařízení s velkým počtem výrobců. Graf s vyznačením všech výrobců by byl nečitelný. 


  \begin{figure}[!h]
    \includegraphics[width=\linewidth]{piechart.png}

  \end{figure}


  \newpage
  \section*{Graf č.3: Počet nalezených zařízení}
  Tento graf zobrazuje počty všech zařízení odchycených v daných časových intervalech. Graf zaznamenává počty všech odchycených zařízení, tzn. i těch, u kterých se nepodařilo zjistit výrobce. Osa X zobrazuje časové intervaly a osa Y zobrazuje počty všech odchycených zařízení v kusech. V případě tohoto grafu je z estetických důvodu prováděna mírná korekce výsledné křivky.
  \vspace{0.5cm}


  \begin{figure}[!h]
    \includegraphics[width=\linewidth]{numberofdevice.png}

  \end{figure}


  \newpage
  \section*{Graf č.4: Počet odchycených WiFi rámců jednotlivých MAC adres}
  Tento graf zobrazuje počty odchycených WiFi rámců jednotlivých MAC adres. Na ose X se zobrazuje jednotlivé odchycené MAC adresy zařízení a na ose Y se zobrazuje příslušným sloupcem nad MAC adresou počet jejich odchycených WiFi rámců v kusech. V případě, že se Vám vygeneroval graf znázorňující počet WiFi rámců u šedesáti nejčastěji odchycených MAC adres, byly bakalářskou aplikací odchyceny WiFi rámců velkého početu různých MAC adres. Graf s vyznačením všech MAC adres by byl nečitelný. 


  \begin{figure}[!h]
    \includegraphics[width=\linewidth]{bargraph.png}

  \end{figure}
Generováno dne \today\

\end{document}